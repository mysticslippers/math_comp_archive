\documentclass[12pt]{report}

\usepackage{cmap}
\usepackage[T1,T2A]{fontenc}
\usepackage[utf8]{inputenc}
\usepackage[english, russian]{babel}
\usepackage{amssymb}
\usepackage{amsmath}
\usepackage{amsthm}
\usepackage{dsfont}
\usepackage{bm}
\usepackage{diagbox}
\usepackage[left=20mm,right=10mm,top=20mm,bottom=20mm,bindingoffset=2mm]{geometry}
\usepackage{indentfirst}
\usepackage[utf8]{inputenc}
\usepackage{float}
\usepackage[hidelinks]{hyperref}
\usepackage{graphicx}
\usepackage{xcolor}
\usepackage{listings}
\usepackage{minted}

\DeclareMathOperator{\N}{\mathbb{N}}
\DeclareMathOperator{\R}{\mathbb{R}}
\DeclareMathOperator{\Z}{\mathbb{Z}}
\DeclareMathOperator{\CC}{\mathbb{C}}
\DeclareMathOperator{\PP}{\mathrm{P}}
\DeclareMathOperator{\Expec}{\mathrm{E}}
\DeclareMathOperator{\Var}{\mathrm{Var}}
\DeclareMathOperator{\Cov}{\mathrm{Cov}}
\DeclareMathOperator{\asConv}{\xrightarrow{a.s.}}
\DeclareMathOperator{\LpConv}{\xrightarrow{Lp}}
\DeclareMathOperator{\pConv}{\xrightarrow{p}}
\DeclareMathOperator{\dConv}{\xrightarrow{d}}

\hypersetup{
    colorlinks=true,
    linkcolor=blue,
    citecolor=blue,
    urlcolor=blue
}

\lstset{language=Python, extendedchars=\true}

\lstdefinestyle{pythonstyle}{
    language=Python,
    backgroundcolor=\color{lightgray},
    commentstyle=\color{green},
    keywordstyle=\color{blue},
    stringstyle=\color{red},
    basicstyle=\ttfamily,
    frame=single,
    breaklines=true
}

\addto\captionsrussian{\renewcommand{\refname}{Список использованных источников}}

\begin{document}

\begin{titlepage}
    \begin{center}
        \large{Федеральное государственное автономное образовательное учреждение высшего образования <<Национальный исследовательский университет ИТМО>>}
    \end{center}
    
    \vspace{15em}
    
    \begin{center}
        \huge{\textbf{Лабораторная работа №1}} \\
        \large{По дисциплине <<Вычислительная математика>>} \\
        \large{Вариант №8}
    \end{center}
    
    \vspace{5em}
    
    \begin{flushright}
        \textit{\large{Выполнил:}} \\
        \large{Студент группы P3206} \\
        \large{Михайлов Дмитрий} \\
        \large{Андреевич} \\
        \textit{\large{Преподаватель:}} \\
        \large{Малышева Татьяна} \\
        \large{Алексеевна}
    \end{flushright}

    \vspace{2cm}

    \begin{figure}[h]
        \centering
        \includegraphics[width=0.5\linewidth]{image.png}
    \end{figure}
    
    \begin{center}
        Санкт-Петербург \\
        2025 год
    \end{center}
\end{titlepage}

\tableofcontents
\newpage

\addcontentsline{toc}{section}{Цель работы}
\section*{Цель работы}
Изучить численные методы решения систем линейных алгебраических уравнений и реализовать один из них средствами программирования.

\addcontentsline{toc}{section}{Описание метода}
\section*{Описание метода}
Метод Гаусса с выбором главного элемента по столбцам.
Схема с выбором главного элемента является одной из модификаций метода Гаусса. Идеей является такая перестановка уравнений, чтобы на k-ом шаге исключения ведущим элементом $a_{ii}$ оказывался наибольший по модулю элемент k-го столбца.

\addcontentsline{toc}{section}{Листинг программы}
\section*{Листинг программы}
\href{https://github.com/mysticslippers/math_comp_archive/blob/main/Lab1_comp_math/code/main.py}{Ссылка на репозиторий с кодом.}
\newpage

\addcontentsline{toc}{section}{Блок-схема метода}
\section*{Блок-схема метода}

\begin{figure}[h]
        \centering
        \includegraphics[width=1\linewidth]{Рисунок1.png}
\end{figure}
\newpage

\addcontentsline{toc}{section}{Пример работы программы}
\section*{Пример работы программы}

\begin{figure}[h]
        \centering
        \includegraphics[scale=0.55]{Рисунок5.png}
        \includegraphics[scale=0.55]{Рисунок4.png}
\end{figure}

\addcontentsline{toc}{section}{Вывод}
\section*{Вывод}
В результате выполнения данной лабораторной работой я познакомился с численными методами решения математических задач на примере систем алгебраических уравнений, реализовав на языке программирования Python метод Гаусса с выбором главного элемента по столбцам.
\end{document}
