\documentclass[12pt]{report}

\usepackage{cmap}
\usepackage[T1,T2A]{fontenc}
\usepackage[utf8]{inputenc}
\usepackage[english, russian]{babel}
\usepackage{amssymb}
\usepackage{amsmath}
\usepackage{amsthm}
\usepackage{dsfont}
\usepackage{bm}
\usepackage{diagbox}
\usepackage[left=20mm,right=10mm,top=20mm,bottom=20mm,bindingoffset=2mm]{geometry}
\usepackage{indentfirst}
\usepackage[utf8]{inputenc}
\usepackage{float}
\usepackage[hidelinks]{hyperref}
\usepackage{graphicx}
\usepackage{xcolor}
\usepackage{listings}
\usepackage{minted}

\DeclareMathOperator{\N}{\mathbb{N}}
\DeclareMathOperator{\R}{\mathbb{R}}
\DeclareMathOperator{\Z}{\mathbb{Z}}
\DeclareMathOperator{\CC}{\mathbb{C}}
\DeclareMathOperator{\PP}{\mathrm{P}}
\DeclareMathOperator{\Expec}{\mathrm{E}}
\DeclareMathOperator{\Var}{\mathrm{Var}}
\DeclareMathOperator{\Cov}{\mathrm{Cov}}
\DeclareMathOperator{\asConv}{\xrightarrow{a.s.}}
\DeclareMathOperator{\LpConv}{\xrightarrow{Lp}}
\DeclareMathOperator{\pConv}{\xrightarrow{p}}
\DeclareMathOperator{\dConv}{\xrightarrow{d}}

\hypersetup{
    colorlinks=true,
    linkcolor=blue,
    citecolor=blue,
    urlcolor=blue
}

\lstset{language=Python, extendedchars=\true}

\lstdefinestyle{pythonstyle}{
    language=Python,
    backgroundcolor=\color{lightgray},
    commentstyle=\color{green},
    keywordstyle=\color{blue},
    stringstyle=\color{red},
    basicstyle=\ttfamily,
    frame=single,
    breaklines=true
}

\addto\captionsrussian{\renewcommand{\refname}{Список использованных источников}}

\begin{document}

\begin{titlepage}
    \begin{center}
        \large{Федеральное государственное автономное образовательное учреждение высшего образования <<Национальный исследовательский университет ИТМО>>}
    \end{center}
    
    \vspace{15em}
    
    \begin{center}
        \huge{\textbf{Лабораторная работа №1}} \\
        \large{По дисциплине <<Вычислительная математика>>} \\
        \large{Вариант №8}
    \end{center}
    
    \vspace{5em}
    
    \begin{flushright}
        \textit{\large{Выполнил:}} \\
        \large{Студент группы P3206} \\
        \large{Михайлов Дмитрий} \\
        \large{Андреевич} \\
        \textit{\large{Преподаватель:}} \\
        \large{Малышева Татьяна} \\
        \large{Алексеевна}
    \end{flushright}

    \vspace{2cm}

    \begin{figure}[h]
        \centering
        \includegraphics[width=0.5\linewidth]{image.png}
    \end{figure}
    
    \begin{center}
        Санкт-Петербург \\
        2025 год
    \end{center}
\end{titlepage}

\tableofcontents
\newpage

\addcontentsline{toc}{section}{Цель работы}
\section*{Цель работы}
Изучить численные методы решения систем линейных алгебраических уравнений и реализовать один из них средствами программирования.

\addcontentsline{toc}{section}{Описание метода}
\section*{Описание метода}
Метод Гаусса с выбором главного элемента по столбцам.
Схема с выбором главного элемента является одной из модификаций метода Гаусса. Идеей является такая перестановка уравнений, чтобы на k-ом шаге исключения ведущим элементом $a_{ii}$ оказывался наибольший по модулю элемент k-го столбца.

\addcontentsline{toc}{section}{Листинг программы}
\section*{Листинг программы}
\begin{lstlisting}[style=pythonstyle]
def solveMinor(matrix, i, j):
    """ Найти минор элемента матрицы """
    n = len(matrix)
    return [[matrix[row][col] for col in range(n) if col != j] for row in range(n) if row != i]


def solveDet(matrix):
    """ Найти определитель матрицы """
    n = len(matrix)
    if n == 1:
        return matrix[0][0]
    det = 0
    sgn = 1
    for j in range(n):
        det += sgn * matrix[0][j] * solveDet(solveMinor(matrix, 0, j))
        sgn *= -1
    return det


def solve(matrix):
    """ Метод Гаусса с выбором главного элемента по столбцам """
    n = len(matrix)
    det = solveDet([matrix[i][:n] for i in range(n)])
    if det == 0:
        return None

    # Прямой ход
    for i in range(n - 1):
        # Поиск максимального элемента в столбце
        max_i = i
        for m in range(i + 1, n):
            if abs(matrix[m][i]) > abs(matrix[max_i][i]):
                max_i = m

        # Перестановка строк
        if max_i != i:
            for j in range(n + 1):
                matrix[i][j], matrix[max_i][j] = matrix[max_i][j], matrix[i][j]

        # Исключение i-того неизвестного
        for k in range(i + 1, n):
            coef = matrix[k][i] / matrix[i][i]
            for j in range(i, n + 1):
                matrix[k][j] -= coef * matrix[i][j]

    reduced_matrix = matrix[:]

    # Обратный ход
    roots = [0] * n
    for i in range(n - 1, -1, -1):
        s_part = 0
        for j in range(i + 1, n):
            s_part += matrix[i][j] * roots[j]
        roots[i] = (matrix[i][n] - s_part) / matrix[i][i]

    # Вычисление невязок
    residuals = [0] * n
    for i in range(n):
        s_part = 0
        for j in range(n):
            s_part += matrix[i][j] * roots[j]
        residuals[i] = s_part - matrix[i][n]

    return det, reduced_matrix, roots, residuals


def main():
    print("\t\tМетод Гаусса с выбором главного элемента по столбцам")
    print("\nВзять коэффициенты из файла или ввести с клавиатуры? (+/-)")

    method = input(">>> ")
    while (method != '+') and (method != '-'):
        print("Введите '+' или '-' для выбора способа ввода.")
        method = input(">>> ")

    if method == '+':
        matrix = readFile()
    else:
        matrix = readConsole()

    if matrix is None:
        print("При считывании коэффициентов матрицы произошла ошибка!")
        return

    answer = solve(matrix[:])
    if answer is None:
        print("\nМатрица является несовместной.")
        return
    det, reduced_matrix, roots, residuals = answer

    print("\nОпределитель:")
    print(det)

    print("\nПреобразованная матрица:")
    for row in reduced_matrix:
        for col in row:
            print('{:10}'.format(round(col, 3)), end='')
        print()

    print("\nВектор неизвестных:")
    for root in roots:
        print('  ' + str(root))

    print("\nВектор невязок:")
    for residual in residuals:
        print('  ' + str(residual))

    input("\n\nНажмите Enter, чтобы выйти.")


main()

\end{lstlisting}
\newpage

\addcontentsline{toc}{section}{Блок-схема метода}
\section*{Блок-схема метода}

\begin{figure}[h]
        \centering
        \includegraphics[width=1\linewidth]{Рисунок1.png}
\end{figure}
\newpage

\addcontentsline{toc}{section}{Пример работы программы}
\section*{Пример работы программы}

\begin{figure}[h]
        \centering
        \includegraphics[scale=0.55]{Рисунок5.png}
        \includegraphics[scale=0.55]{Рисунок4.png}
\end{figure}

\addcontentsline{toc}{section}{Вывод}
\section*{Вывод}
В результате выполнения данной лабораторной работой я познакомился с численными методами решения математических задач на примере систем алгебраических уравнений, реализовав на языке программирования Python метод Гаусса с выбором главного элемента по столбцам.
\end{document}
