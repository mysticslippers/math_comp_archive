\documentclass[12pt]{report}

\usepackage{cmap}
\usepackage[T1,T2A]{fontenc}
\usepackage[utf8]{inputenc}
\usepackage[english, russian]{babel}
\usepackage{amssymb}
\usepackage{amsmath}
\usepackage{amsthm}
\usepackage{dsfont}
\usepackage{bm}
\usepackage{diagbox}
\usepackage[left=20mm,right=10mm,top=20mm,bottom=20mm,bindingoffset=2mm]{geometry}
\usepackage{indentfirst}
\usepackage[utf8]{inputenc}
\usepackage{float}
\usepackage[hidelinks]{hyperref}
\usepackage{graphicx}
\usepackage{xcolor}
\usepackage{listings}
\usepackage{minted}

\DeclareMathOperator{\N}{\mathbb{N}}
\DeclareMathOperator{\R}{\mathbb{R}}
\DeclareMathOperator{\Z}{\mathbb{Z}}
\DeclareMathOperator{\CC}{\mathbb{C}}
\DeclareMathOperator{\PP}{\mathrm{P}}
\DeclareMathOperator{\Expec}{\mathrm{E}}
\DeclareMathOperator{\Var}{\mathrm{Var}}
\DeclareMathOperator{\Cov}{\mathrm{Cov}}
\DeclareMathOperator{\asConv}{\xrightarrow{a.s.}}
\DeclareMathOperator{\LpConv}{\xrightarrow{Lp}}
\DeclareMathOperator{\pConv}{\xrightarrow{p}}
\DeclareMathOperator{\dConv}{\xrightarrow{d}}

\hypersetup{
	colorlinks=true,
	linkcolor=blue,
	citecolor=blue,
	urlcolor=blue
}

\lstset{language=Python, extendedchars=\true}

\lstdefinestyle{pythonstyle}{
	language=Python,
	backgroundcolor=\color{lightgray},
	commentstyle=\color{green},
	keywordstyle=\color{blue},
	stringstyle=\color{red},
	basicstyle=\ttfamily,
	frame=single,
	breaklines=true
}

\addto\captionsrussian{\renewcommand{\refname}{Список использованных источников}}

\begin{document}
	
	\begin{titlepage}
		\begin{center}
			\large{Федеральное государственное автономное образовательное учреждение высшего образования <<Национальный исследовательский университет ИТМО>>}
		\end{center}
		
		\vspace{15em}
		
		\begin{center}
			\huge{\textbf{Лабораторная работа №2}} \\
			\large{По дисциплине <<Вычислительная математика>>} \\
			\large{Вариант №8}
		\end{center}
		
		\vspace{5em}
		
		\begin{flushright}
			\textit{\large{Выполнил:}} \\
			\large{Студент группы P3206} \\
			\large{Михайлов Дмитрий} \\
			\large{Андреевич} \\
			\textit{\large{Преподаватель:}} \\
			\large{Малышева Татьяна} \\
			\large{Алексеевна}
		\end{flushright}
		
		\vspace{2cm}
		
		\begin{figure}[h]
			\centering
			\includegraphics[width=0.5\linewidth]{image.png}
		\end{figure}
		
		\begin{center}
			Санкт-Петербург \\
			2025 год
		\end{center}
	\end{titlepage}
	
	\tableofcontents
	\newpage
	
	\addcontentsline{toc}{section}{Цель работы}
	\section*{Цель работы}
	Изучить численные методы решения нелинейных уравнений и их
	систем, найти корни заданного нелинейного уравнения/системы нелинейных уравнений, выполнить программную реализацию методов.
	
	\addcontentsline{toc}{section}{Ход работы}
	\section*{Ход работы}
	\begin{table}[h!]
		\centering
		\begin{tabular}{|c|c|c|c|c|}
			\hline
			№ шага & $x_k$ & $x_{k+1}$ & $f(x_{k+1})$ & $|x_k - x_{k+1}|$ \\ \hline
			1 & 3.000 & 2.788 & 1.018 & 0.212 \\ \hline
			2 & 2.788 & 2.697 & 0.061 & 0.091 \\ \hline
			3 & 2.697 & 2.682 & 0.003 & 0.015 \\ \hline
			4 & 2.682 & 2.680 & -0.0005 & 0.002 \\ \hline
		\end{tabular}
		\caption{Уточнение крайнего правого корня методом простой итерации}
		\label{tab:1}
	\end{table}

	\begin{table}[h!]
		\centering
		\begin{tabular}{|c|c|c|c|c|c|c|c|}
			\hline
			№ шага & $a$ & $b$ & $x_k$ & $f(a)$ & $f(b)$ & $f(x)$ & $|x_k - x_{k+1}|$ \\ \hline
			
			1 & -3 & 3 & -1.83 & -13.55 & 56.93 & -9.039 & 2.415\\ \hline
			
			2 & -1.83 & 3 & 0.585 & -9.039 & 56.93 & -5.957 & 1.2075 \\ \hline
			
			3 & 0.585 & 3 & 1.7925 & -5.957 & 56.93 & -2.212 & 0.604 \\ \hline
			
			4 & 1.7925 & 3 & 2.39625 & -2.212 & 56.93 & 3.913 & 0.302 \\ \hline
			
			5 & 1.7925 & 2.39625 & 2.094375 & -2.212 & 3.913 & -0.634 & 0.151 \\ \hline
			
			6 & 2.094375 & 2.39625 & 2.2453125 & -0.634 & -2.212 & 1.614 & 0.075 \\ \hline
			
			7 & 2.094375 & 2.2453125 & 2.16984375 & -0.634 & 1.614 & 0.307 & 0.038 \\ \hline
			
			8 & 2.094375 & 2.16984375 & 2.132109375 & -0.634 & 0.307 & -0.147 & 0.019 \\ \hline
			
			9 & 2.132109375 & 2.16984375 & 2.1509765625 & -0.147 & 0.307 & 0.077 & 0.009 \\ \hline
			
			10 & 2.132109375 & 2.1509765625 & 2.14104296875 & -0.147 & 0.077 & -0.035 & - \\ \hline
		\end{tabular}
		\caption{Уточнение крайнего левого корня методом хорд}
		\label{tab:2}
	\end{table}	
	
	\begin{table}[h!]
		\centering
		\begin{tabular}{|c|c|c|c|c|c|}
			\hline
			№ шага & $x_{k-1}$ & $x_k$ & $x_{k+1}$ & $f(x_{k+1})$ & $|x_k - x_{k+1}|$ \\ \hline
			
			1 & -3 & 3 & -1.839 & -2.992 & 4.839 \\ \hline
			
			2 & 3 & -1.839 & -2.079 & 1.030 & 0.240 \\ \hline
			
			3 & -1.839 & -2.079 & -2.161 & 0.082 & 0.082 \\ \hline
			
			4 & -2.079 & -2.161 & -2.155 & -0.021 & 0.006 \\ \hline
		\end{tabular}
		\caption{Уточнение центрального корня методом секущих}
		\label{tab:3}
	\end{table}	
	
	\begin{figure}[h]
		\centering
		\includegraphics[width=0.55\linewidth]{Рисунок6}
		\label{fig:1}
		\caption{График функции 3 * $x^3$ +1.7 * $x^2$ - 15.42 * x + 6.89.}
	\end{figure}
	\newpage
	
	\addcontentsline{toc}{section}{Блок-схемы используемых методов}
	\section*{Блок-схемы используемых методов}
	
	\begin{figure}[H]
		\centering
		\includegraphics[width=0.4\linewidth]{Рисунок3.png}
		\caption{Блок-схема метода простой итерации}
	\end{figure}
	
	\begin{figure}[H]
		\centering
		\includegraphics[width=0.4\linewidth]{Рисунок1.png}
		\caption{Блок-схема метода хорд}
	\end{figure}
	
	\begin{figure}[H]
		\centering
		\includegraphics[width=0.4\linewidth]{Рисунок2.png}
		\caption{Блок-схема метоа секущих}
	\end{figure}
	
	\addcontentsline{toc}{section}{Листинг программы}
	\section*{Листинг программы}
	\href{https://github.com/mysticslippers/math_comp_archive/blob/main/Lab2_comp_math/code/main.py}{Ссылка на репозиторий с кодом.}
	\newpage
	
	\addcontentsline{toc}{section}{Результат выполнения программы}
	\section*{Результат выполнения программы}
	
	\begin{figure}[h]
		\includegraphics[width=0.7\linewidth]{Рисунок4}
		\label{fig:4}
		\caption{Пример выполнения программы.}
	\end{figure}
	
\begin{figure}[h]
	\includegraphics[scale=0.4]{Рисунок5}
	\caption{Пример выполнения программы.}
	\label{fig:6}
\end{figure}
	
	\addcontentsline{toc}{section}{Вывод}
	\section*{Вывод}
	В результате выполнения данной лабораторной работой я познакомился с численными методами решения нелинейных уравнений и реализовал метод хорд, метод секущих и метод простой итерации на языке программирования Python, закрепив знания.
\end{document}
