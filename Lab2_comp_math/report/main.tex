\documentclass[12pt]{report}

\usepackage{cmap}
\usepackage[T1,T2A]{fontenc}
\usepackage[utf8]{inputenc}
\usepackage[english, russian]{babel}
\usepackage{amssymb}
\usepackage{amsmath}
\usepackage{amsthm}
\usepackage{dsfont}
\usepackage{bm}
\usepackage{diagbox}
\usepackage[left=20mm,right=10mm,top=20mm,bottom=20mm,bindingoffset=2mm]{geometry}
\usepackage{indentfirst}
\usepackage[utf8]{inputenc}
\usepackage{float}
\usepackage[hidelinks]{hyperref}
\usepackage{graphicx}
\usepackage{xcolor}
\usepackage{listings}
\usepackage{minted}

\DeclareMathOperator{\N}{\mathbb{N}}
\DeclareMathOperator{\R}{\mathbb{R}}
\DeclareMathOperator{\Z}{\mathbb{Z}}
\DeclareMathOperator{\CC}{\mathbb{C}}
\DeclareMathOperator{\PP}{\mathrm{P}}
\DeclareMathOperator{\Expec}{\mathrm{E}}
\DeclareMathOperator{\Var}{\mathrm{Var}}
\DeclareMathOperator{\Cov}{\mathrm{Cov}}
\DeclareMathOperator{\asConv}{\xrightarrow{a.s.}}
\DeclareMathOperator{\LpConv}{\xrightarrow{Lp}}
\DeclareMathOperator{\pConv}{\xrightarrow{p}}
\DeclareMathOperator{\dConv}{\xrightarrow{d}}

\hypersetup{
	colorlinks=true,
	linkcolor=blue,
	citecolor=blue,
	urlcolor=blue
}

\lstset{language=Python, extendedchars=\true}

\lstdefinestyle{pythonstyle}{
	language=Python,
	backgroundcolor=\color{lightgray},
	commentstyle=\color{green},
	keywordstyle=\color{blue},
	stringstyle=\color{red},
	basicstyle=\ttfamily,
	frame=single,
	breaklines=true
}

\addto\captionsrussian{\renewcommand{\refname}{Список использованных источников}}

\begin{document}
	
	\begin{titlepage}
		\begin{center}
			\large{Федеральное государственное автономное образовательное учреждение высшего образования <<Национальный исследовательский университет ИТМО>>}
		\end{center}
		
		\vspace{15em}
		
		\begin{center}
			\huge{\textbf{Лабораторная работа №2}} \\
			\large{По дисциплине <<Вычислительная математика>>} \\
			\large{Вариант №8}
		\end{center}
		
		\vspace{5em}
		
		\begin{flushright}
			\textit{\large{Выполнил:}} \\
			\large{Студент группы P3206} \\
			\large{Михайлов Дмитрий} \\
			\large{Андреевич} \\
			\textit{\large{Преподаватель:}} \\
			\large{Малышева Татьяна} \\
			\large{Алексеевна}
		\end{flushright}
		
		\vspace{2cm}
		
		\begin{figure}[h]
			\centering
			\includegraphics[width=0.5\linewidth]{image.png}
		\end{figure}
		
		\begin{center}
			Санкт-Петербург \\
			2025 год
		\end{center}
	\end{titlepage}
	
	\tableofcontents
	\newpage
	
	\addcontentsline{toc}{section}{Цель работы}
	\section*{Цель работы}
	Изучить численные методы решения нелинейных уравнений и их
	систем, найти корни заданного нелинейного уравнения/системы нелинейных уравнений, выполнить программную реализацию методов.
	
	\addcontentsline{toc}{section}{Ход работы}
	\section*{Ход работы}
	\begin{table}[h!]
		\centering
		\begin{tabular}{|c|c|c|c|c|c|}
			\hline
			№ шага & $x_k$ & $f(x_k)$ & $x_{k+1}$ & $\phi(x_k)$ & $|x_k - x_{k+1}|$ \\ \hline
			 &  &  &  &  & \\ \hline
			 &  &  &  &  & \\ \hline
			 &  &  &  &  & \\ \hline
			 &  &  &  &  & \\ \hline
			 &  &  &  &  & \\ \hline
			 &  &  &  &  & \\ \hline
			 &  &  &  &  & \\ \hline
			 &  &  &  &  & \\ \hline
		\end{tabular}
		\caption{Уточнение крайнего правого корня методом простой итерации}
		\label{tab:1}
	\end{table}

	\begin{table}[h!]
		\centering
		\begin{tabular}{|c|c|c|c|c|c|c|c|}
			\hline
			№ шага & $a$ & $b$ & $x_k$ & $f(a)$ & $f(b)$ & $f(x)$ & $|x_k - x_{k+1}|$ \\ \hline
			&  &  &  &  &  &  & \\ \hline
			&  &  &  &  &  &  & \\ \hline
			&  &  &  &  &  &  & \\ \hline
			&  &  &  &  &  &  & \\ \hline
			&  &  &  &  &  &  & \\ \hline
			&  &  &  &  &  &  & \\ \hline
			&  &  &  &  &  &  & \\ \hline
			&  &  &  &  &  &  & \\ \hline
		\end{tabular}
		\caption{Уточнение крайнего левого корня методом хорд}
		\label{tab:2}
	\end{table}	
	
	\begin{table}[h!]
		\centering
		\begin{tabular}{|c|c|c|c|c|c|c|c|}
			\hline
			№ шага & $x_{k-1}$ & $f(x_{k-1})$ & $x_k$ & $f(x_k)$ & $x_{k+1}$ & $f(x_{k+1})$ & $|x_k - x_{k+1}|$ \\ \hline
			&  &  &  &  &  &  & \\ \hline
			&  &  &  &  &  &  & \\ \hline
			&  &  &  &  &  &  & \\ \hline
			&  &  &  &  &  &  & \\ \hline
			&  &  &  &  &  &  & \\ \hline
			&  &  &  &  &  &  & \\ \hline
			&  &  &  &  &  &  & \\ \hline
			&  &  &  &  &  &  & \\ \hline
		\end{tabular}
		\caption{Уточнение центрального корня методом секущих}
		\label{tab:3}
	\end{table}	
	\newpage
	
	\addcontentsline{toc}{section}{Блок-схемы используемых методов}
	\section*{Блок-схемы используемых методов}
	
	\begin{figure}[htbp]
		\centering
		\includegraphics[width=0.22\linewidth]{Рисунок3.png}
		\caption{Блок-схема метода простой итерации}
		
		\includegraphics[width=0.22\linewidth]{Рисунок1.png}
		\caption{Блок-схема метода хорд}
		
		\includegraphics[width=0.22\linewidth]{Рисунок2.png}
		\caption{Блок-схема метода секущих}
	\end{figure}
	
	\addcontentsline{toc}{section}{Листинг программы}
	\section*{Листинг программы}
	
	\addcontentsline{toc}{section}{Результат выполнения программы}
	\section*{Результат выполнения программы}
	
	\addcontentsline{toc}{section}{Вывод}
	\section*{Вывод}
	В результате выполнения данной лабораторной работой я познакомился с численными методами решения нелинейных уравнений и реализовал метод хорд, метод секущих и метод простой итерации на языке программирования Python, закрепив знания.
\end{document}
